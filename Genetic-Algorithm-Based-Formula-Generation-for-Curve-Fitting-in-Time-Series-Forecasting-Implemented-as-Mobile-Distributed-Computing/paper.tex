\documentclass[graybox]{svmult}

%\usepackage{mathptmx}
%\usepackage{helvet}
%\usepackage{courier}
%\usepackage{type1cm}
%
%\usepackage{makeidx}
%\usepackage{graphicx}
%
%\usepackage{multicol}
%\usepackage[bottom]{footmisc}

\makeindex

\begin{document}

\title*{Genetic Algorithm Based Formula Generation for Curve Fitting in Time Series Forecasting Implemented as Mobile Distributed Computing}
\titlerunning{GA Formula Generation for CF in TSF Implemented as MDC}

\author{Rumen Ketipov, Georgi Kostadinov, Plamen Petrov, Iliyan Zankinski, Todor Balabanov\textsuperscript{0000-0003-3139-069X}}
\authorrunning{R. Ketipov et al.}

\institute{
	Rumen Ketipov \email{rketipov@iit.bas.bg}
\and 
	Georgi Kostadinov \email{g.kostadinov@iit.bas.bg}
\and 
	Plamen Petrov \email{p.petrov@iit.bas.bg}
\and 
	Iliyan Zankinski \email{iliyan@hsi.iccs.bas.bg}
\and 
	Todor Balabanov \email{todorb@iinf.bas.bg} 
\at 
	Institute of Information and Communication Technologies - Bulgarian Academy of Sciences, acad. Georgi Bonchev Str, block 2, 1113 Sofia, Bulgaria}

\maketitle

\abstract*{Times series forecasting has many important real life applications. Such forecasting is widely used in statistics, signal processing, pattern recognition, econometrics, mathematical finance, weather forecasting, earthquake prediction, electroencephalography, control engineering, astronomy, communications engineering and in any applied mathematics field where temporal measurements are done. 
%
In last few decades time series forecasting receives a lot of attention from the researchers in the machine learning domain. Many different forecasting models are developed with the usage of different prediction approaches. Artificial neural networks are a bright example of such forecasting technique. The main goal is learning of data dependency between past and future values of the time series when artificial neural network is used. If the weights of the artificial neural networks are taken as coefficients of a complex polynomial the forecasting can be presented as curve fitting problem. 
%
This research proposes forecasting approach a little bit different than the approach used in the artificial neural networks. Set of mathematical formulas are presented as expression trees in a genetic algorithm population. The goal in this genetic algorithm based optimization is searching of a mathematical expression which can provide the best curve fitting formula according time series values. Because of the genetic algorithms' extremely high degree of parallelism possibilities calculations in this research are organized as distributed computing solutions on a mobile devices with Android operating system.}

\abstract{Times series forecasting has many important real life applications. Such forecasting is widely used in statistics, signal processing, pattern recognition, econometrics, mathematical finance, weather forecasting, earthquake prediction, electroencephalography, control engineering, astronomy, communications engineering and in any applied mathematics field where temporal measurements are done. 
%
In last few decades time series forecasting receives a lot of attention from the researchers in the machine learning domain. Many different forecasting models are developed with the usage of different prediction approaches. Artificial neural networks are a bright example of such forecasting technique. The main goal is learning of data dependency between past and future values of the time series when artificial neural network is used. If the weights of the artificial neural networks are taken as coefficients of a complex polynomial the forecasting can be presented as curve fitting problem. 
%
This research proposes forecasting approach a little bit different than the approach used in the artificial neural networks. Set of mathematical formulas are presented as expression trees in a genetic algorithm population. The goal in this genetic algorithm based optimization is searching of a mathematical expression which can provide the best curve fitting formula according time series values. Because of the genetic algorithms' extremely high degree of parallelism possibilities calculations in this research are organized as distributed computing solutions on a mobile devices with Android operating system.}

\section{Section Heading}
\label{sec:1}

\section{Section Heading}
\label{sec:2}

\end{document}
