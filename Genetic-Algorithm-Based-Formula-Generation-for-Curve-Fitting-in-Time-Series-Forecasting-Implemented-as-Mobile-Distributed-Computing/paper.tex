\documentclass[graybox]{svmult}

\usepackage{mathptmx}
\usepackage{helvet}
\usepackage{courier}
\usepackage{type1cm}
\usepackage{makeidx}
\usepackage{graphicx}
\usepackage{multicol}
\usepackage[bottom]{footmisc}

\makeindex

\begin{document}

\title*{Genetic Algorithm Based Formula Generation for Curve Fitting in Time Series Forecasting Implemented as Mobile Distributed Computing}
\titlerunning{GA Formula Generation for CF in TSF Implemented as MDC}

\author{Rumen Ketipov, Georgi Kostadinov, Plamen Petrov,  \\ Iliyan Zankinski, Todor Balabanov\textsuperscript{0000-0003-3139-069X}}
\authorrunning{R. Ketipov et al.}

\institute{
	Rumen Ketipov \email{rketipov@iit.bas.bg}
\and 
	Georgi Kostadinov \email{g.kostadinov@iit.bas.bg}
\and 
	Plamen Petrov \email{p.petrov@iit.bas.bg}
\and 
	Iliyan Zankinski \email{iliyan@hsi.iccs.bas.bg}
\and 
	Todor Balabanov \email{todorb@iinf.bas.bg} 
\at 
	Institute of Information and Communication Technologies - Bulgarian Academy of Sciences, acad. Georgi Bonchev Str, block 2, 1113 Sofia, Bulgaria}

\maketitle

\abstract*{Times series forecasting has many important real life applications. Such forecasting is widely used in statistics, signal processing, pattern recognition, econometrics, mathematical finance, weather forecasting, earthquake prediction, electroencephalography, control engineering, astronomy, communications engineering and in any applied mathematics field where temporal measurements are done. 
\vskip 0.01em
In last few decades time series forecasting receives a lot of attention from the researchers in the machine learning domain. Many different forecasting models are developed with the usage of different prediction approaches. Artificial neural networks are a bright example of such forecasting technique. The main goal is learning of data dependency between past and future values of the time series when artificial neural network is used. If the weights of the artificial neural networks are taken as coefficients of a complex polynomial the forecasting can be presented as curve fitting problem. 
\vskip 0.01em
This research proposes forecasting approach a little bit different than the approach used in the artificial neural networks. Set of mathematical formulas are presented as expression trees in a genetic algorithm population. The goal in this genetic algorithm based optimization is searching of a mathematical expression which can provide the best curve fitting formula according time series values. Because of the genetic algorithms' extremely high degree of parallelism possibilities calculations in this research are organized as distributed computing solutions on a mobile devices with Android operating system.}

\abstract{Times series forecasting has many important real life applications. Such forecasting is widely used in statistics, signal processing, pattern recognition, econometrics, mathematical finance, weather forecasting, earthquake prediction, electroencephalography, control engineering, astronomy, communications engineering and in any applied mathematics field where temporal measurements are done. 
\vskip 0.01em
In last few decades time series forecasting receives a lot of attention from the researchers in the machine learning domain. Many different forecasting models are developed with the usage of different prediction approaches. Artificial neural networks are a bright example of such forecasting technique. The main goal is learning of data dependency between past and future values of the time series when artificial neural network is used. If the weights of the artificial neural networks are taken as coefficients of a complex polynomial the forecasting can be presented as curve fitting problem. 
\vskip 0.01em
This research proposes forecasting approach a little bit different than the approach used in the artificial neural networks. Set of mathematical formulas are presented as expression trees in a genetic algorithm population. The goal in this genetic algorithm based optimization is searching of a mathematical expression which can provide the best curve fitting formula according time series values. Because of the genetic algorithms' extremely high degree of parallelism possibilities calculations in this research are organized as distributed computing solutions on a mobile devices with Android operating system.}

\section{Introduction} \label{Introduction}

Financial time series forecasting is an area of high researchers interest \cite{nava01} from many decades. Having an accurate forecast in the financial world is crucial for many important decision makings. Time series are ordered measurements of particular variable done in a temporal manner. In most cases values are measured on equal intervals, but it is not a mandatory condition. Time series analysis is applicable in processes with clear repetition pattern. Measurements done in a temporal order are presented as points in a two-dimensional space. For visualization purposes in many cases these points are connected with straight lines, which is a simplified form of linear interpolation for the values between two neighboring measurements. With such organization of points in two-dimensional space many different curves can be proposed for some generalization of the data nature. Linear regression for example can be used for trend estimation. In this case parameters of line equation are calculated such as that this line is the closest line for the given points. Another much more complicated generalization can be the Lagrange polynomial. In this case the parameters of a polynomial of lowest degree that assumes at each value from X the corresponding value on Y (curve coincide at each point) are estimated. Theoretically infinite number of curves can satisfy the condition to pass across finite number of points in two-dimensional space. In the same context if calculations done in an artificial neural network are evaluated as mathematical expression it will give rough description of an equation in two-dimensional space. The general advantage of artificial neural networks is that they are capable to self-adjust coefficients in this equation. 

\section{Conclusions} \label{Conclusions}

As future research it will be interesting different communication capabilities to be investigated as described in \cite{alexandrov01} in order better genetic algorithms indivuduals exchange to be achieved. Also it will be interesting generalized artificial neural networks \cite{tashev01} to be tested in combination with genetic algorithms formula generation.

\begin{acknowledgement}
This work was funded by Velbazhd Software LLC.
\end{acknowledgement}

\begin{thebibliography}{99}

\bibitem{nava01} Nava, N., Di Matteo, T., Aste, T., \textbf{\textit{Financial Time Series Forecasting Using Empirical Mode Decomposition and Support Vector Regression}}, RISKS, \textbf{6}(1), article 7, 2018.

\bibitem{zhang01} Zhang, R., Tao, J., \textbf{\textit{A Nonlinear Fuzzy Neural Network Modeling Approach Using an Improved Genetic Algorithm}}, IEEE Transactions on Industrial Electronics, \textbf{65}(7), p. 5882--5892, 2018.

\bibitem{kapanova01} Kapanova, K., Dimov, I., Sellier, J.M., \textbf{\textit{A genetic approach to automatic neural network architecture optimization}}, Neural Computing and Applications, Springer London, \textbf{29}(5), p. 1481--1492, 2016.

\bibitem{aljarah01} Aljarah, I., Faris, H., Mirjalili, S., \textbf{\textit{Optimizing connection weights in neural networks using the whale optimization algorithm}}, Soft Computing, Springer Berlin Heidelberg, \textbf{22}(1), p. 1--15, 2016.

\bibitem{altinoz01} Altinoz, O.T., Deb, K., \textbf{\textit{Late parallelization and feedback approaches for distributed computation of evolutionary multi-objective optimization algorithms}}, Neural Computing and Applications, Springer London, \textbf{30}(3), p. 723--733, 2016.

\bibitem{bohn01} Bohn, A., Guting, T., Mansmann, T., \textbf{\textit{MoneyBee: A new product to predict stock market developments using artificial intelligence and increased calculation capacitiy}} (in German), et al. Wirtschaftsinf, \textbf{45}(3), p. 325--333, 2003.

\bibitem{tomov01} Tomov, P., Zankinski, I., Barova, M., \textbf{\textit{Mobile Alternative of the MoneyBee Project for Financial Forecasting}}, Proceedings of the Annual University Scientific Conference of the National Military University Vasil Levski, Veliko Tarnovo, p. 1085--1089, 2018.

\bibitem{alexandrov01} Alexandrov, A., \textbf{\textit{Comparative analysis of IEEE 802.15.4 based communication protocols used in wireless intelligent sensor systems}}, Proceedings of the International conference RAM, p. 51--54, 2014.

\bibitem{tashev01} Tashev, T., Hristov, H., \textbf{\textit{Modeling of synthesis of information processes with generalized nets}}, Cybernetics and Information Technologies, \textbf{3}(2), p. 92--104, 2003.

\end{thebibliography}

\end{document}
